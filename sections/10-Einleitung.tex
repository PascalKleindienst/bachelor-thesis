% !TEX root = ../TUCthesis.tex
%****************************************************
\chapter{Einleitung}\label{ch:intro}
%****************************************************

Sammelkartenspiele auch \ac{TCG} genannt sind Spiele bei denen es meist mehrere hunderte Karten verschiedener Seltenheitsstufen gibt. Man kann sich aus den Karten eine Auswahl zusammenstellen und mit anderen Spielern gegeneinander spielen. Einige der bekanntesten Vertreter von Sammelkartenspielen sind etwa \emph{Magic: the Gathering}, \emph{Pokemon}, \emph{Yu-Gi-Oh!} und \emph{Hearthstone: Heroes of Warcraft}. 

%% -> MTGJSON
Die Anzahl an Karten eines \ac{TCG} wie Magic: the Gathering steigt von Jahr zu Jahr an. Bei Magic: the Gathering erscheinen pro Jahr zwei bis vier neue Sets mit durchschnittlich zwischen 200 bis 400 Karten. Die Menge der Karten nimmt also jedes Jahr beständig zu \emph{(aktuell gibt es >15.000 verschiedene Karten und insgesamt >25.000 Karten verschiedener Versionen)}\footnote{Basierend auf http://mtgjson.com}. Mit \url{http://mtgjson.com} gibt es einen open-source Datensatz aller Kartendaten, welcher unter anderem auch in \cite{finkpredicting, perkhounkovfinancial} benutzt wird, so dass dieser auch hier zum Einsatz kommen soll. 

%% Online-Shops -> \cite{finkpredicting} und \cite{johnson2013online}
Online-Shops benötigen oft nur einen Bruchteil der Informationen, die sich auf einer Karte befinden und speichern diese tabellarisch, d.h. in einer relationalen Datenbank \cite{johnson2013online}. Gerade für kleinere Online-Shops eignet sich dieser Ansatz, da hierbei der Wartungsaufwand gering gehalten wird \cite{johnson2013online}. Karten-Suchmaschinen erfordern aber hoch strukturierte Daten, da alle Informationen einer Karte effizient gespeichert, verwaltet und Suchanfragen in Echtzeit beantwortet werden müssen. 

%% Decks, Turniere -> \cite{haumagic}
Ein weiteres Einsatzgebiet ist das verwalten und analysieren von Decks und Turnieren. Turniere werden nach dem Schweizer-System, einer Sonderform des Rundenturniers, gespielt, wodurch sich ein Netzwerk aus Paarungen der verschiedenen Runden ergibt. Aus diesem Grund werden Ergebnisse von Turnieren oftmals der Einfachheit halber nur als Rangfolge gespeichert und nicht jedes Rundendergebnis. Mit letzterem Ansatz ließen sich verschiedene Analysen über die Decks und Spieler, wie  zum Beispiel Matchups, das heißt welche Gewinn- oder Verlustchancen ein Deck gegen ein anderes Deck hat, erstellen. 

%% => graphdb im kommen evtl gut geeignet
%% Vergleich rel <-> Graph: \cite{vicknair2010comparison}, \cite{miller2013graph}, \cite{jaiswal2013comparative}
%%
%% Problematik X wird auch in [..] behandelt, allerdings mit einem Schwerpunkt auf ..., der hier nicht im Vordergrund steht.
%% Der Ansatz in X scheint geeignet, so dass Algorithmus Y naher zu betrachten ist.
%% Das Softwarepaket X stellt nach [..] einen Standard im Bereich ... dar, so dass  dieses auch hier zum Einsatz kommen soll.

