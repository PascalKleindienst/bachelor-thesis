% !TEX root = ../TUCthesis.tex
%****************************************************
\section{Ausgewählter Ansatz und Detaillösungen}\label{ch:ansatz}
%****************************************************

\subsection{Datenbasis}
Als Basis der Kartendaten wird mtgjson\footnote{\url{http://mtgjson.com}} benutzt, welche \ac{MtG} Karten und Editions-Daten als \ac{JSON}-Datei bereitstellt. Die Decks hingegen werden von mtgtop8\footnote{\url{http://mtgtop8.com}} bezogen.

\subsection{Datenbanken}
Als Graph-Datenbank wird \emph{Neo4j} Version 3.1 eingesetzt, da diese eine hoch skalierbare Datenbank ist, welche auf allen gängigen Betriebssystemen läuft \cite{goel2015neo4j}. Des Weiteren besitzt Neo4j mit \emph{Cypher} eine ausdrucksstarke Abfragesprache. Ein Vorteil von Cypher ist, dass ab Neo4j 2.1 der Import aus \ac{CSV}-Dateien unterstützt wird \cite{panzarino2014cypher}. 

Als \ac{RDBMS} wird MariaDB\footnote{\url{https://mariadb.com}} Version 10.1 eingesetzt, welche vom Urheber von MySQL entwickelt wird. MariaDB ist ein Fork von MySQL und gilt als dessen evolutionäre Nachfolger \cite{bartholomew2013getting}.

\subsubsection{Import der Daten}
Da eine große Menge an Daten importiert werden müssen, zum Beispiel über 100.000 Übersetzungen, ist eine schnelle Import-Funktion hilfreich. Die schnellste und einfachste Möglichkeit dafür, welche beide Systeme unterstützen, ist die Daten als \ac{CSV}-Dateien zu importieren. In \autoref{listing:cypher.import.csv} befindet sich ein Kommando, um Übersetzungen nach Neo4j zu importieren und in \autoref{listing:sql.import.csv} das Äquivalent für den Import in MySQL/MariaDB.

\begin{listing}[t]
    \caption{Importieren von Übersetzungen in Neo4j}
    \label{listing:cypher.import.csv}
    \begin{lstlisting}[language=SQL]
USING PERIODIC COMMIT 1000
LOAD CSV WITH HEADERS FROM "file:translations.csv" AS row
CREATE (:Translation { name: row.name });
    
USING PERIODIC COMMIT
LOAD CSV WITH HEADERS FROM "file:%s/translations.csv" AS row
MATCH (c:Card { name: row.card })
MATCH (t:Translation { name: row.name })
CREATE (c)-[:HAS_TRANSLATION { lang: row.language }]->(t);
    \end{lstlisting}
\end{listing}

\begin{listing}[t]
    \caption{Importieren von Übersetzungen in MySQL/MariaDB}
    \label{listing:sql.import.csv}
    \begin{lstlisting}[language=SQL]
LOAD DATA LOCAL INFILE 'translations.csv' INTO TABLE `translations`
CHARACTER SET UTF8
FIELDS TERMINATED BY ','
ENCLOSED BY '"'
LINES TERMINATED BY '\r\n'
IGNORE 1 LINES
SET (@card, name, lang, card_id);
    \end{lstlisting}
\end{listing}

\subsection{Programmiersprache}
Als Programmiersprache wird \emph{PyPy}\footnote{\url{http://pypy.org}}, eine alternative \emph{Python} Implementierung, die eine hohe Kompatibilität zu Python hat, verwendet. Die Vorteile von PyPy sind der integrierte \ac{JIT} Compiler, wodurch sich eine schnellere Ausführungszeit ergibt, und der geringere Speicherverbrauch \cite{doglio2015mastering}. Als Version wird \emph{PyPy2.7 v5.6.0} verwendet, welches der Python Version 2.7 entspricht.

\subsection{Betriebssystem}
Die Tests wurden auf einem Ubuntu 14.04 Server mit 5120 MB Arbeitsspeicher und 2 CPU Kernen getestet. Die zur Verfügung stehenden CPU Kerne stammen von einem AMD FX-4100 Prozessor, welcher mit 3.5GHz getaktet ist.

\subsection{Schweizer System}
Aufgrund der Tatsache, dass keine Turnier Daten zur Verfügung stehen, müssen die Turniere nach dem Schweizer System simuliert werden, um einen geeigneten Datensatz zu erstellen. Als Grundlage hierfür wird das Projekt \emph{Swiss Tournament Pairing System - Relational Databases}\footnote{\url{https://benjaminbrandt.com/relational-databases-final-project/}} von Ben Brandt verwendet. Hierbei handelt es sich um ein Schweizer Turnier System für \ac{MtG}, welches in Python geschrieben ist.

\subsubsection{Profiling}
Mit einem \emph{Profiler} können verschiedene Werte wie \emph{Ausführungszeit} oder \emph{Speicherverbrauch} gemessen werden \cite{doglio2015mastering}. Um vergleichen zu können, wie gut die Datenbanksysteme in den einzelnen Testfällen abschneiden, werden die Ausführungszeit und der Speicherverbrauch gemessen. 

Für die Ausführungszeit wird eine einfache Timer-Klasse\footnote{\url{https://www.huyng.com/posts/python-performance-analysis}} benutzt, welche die Zeit mittels der \verb|datetime| Bibliothek misst. Um den Speicherverbrauch zu messen wird die Python-Bibliothek \verb|psutil| benutzt\footnote{\url{http://fa.bianp.net/blog/2013/different-ways-to-get-memory-consumption-or-lessons-learned-from-memory_profiler}}, da diese eine einfache und plattformübergreifende Methode bietet den Speicherverbrauch auszulesen.
