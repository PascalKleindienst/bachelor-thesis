% !TEX root = ../TUCthesis.tex
%****************************************************
\section{Potentielle Ansätze und Probleme}\label{ch:ansaetze-probleme}
%****************************************************

\subsection{Relationale Datenbanken}
% Ansatz: RDBMS da etabliert -> gut für suche
% Beispiel: komplexe SQL Abfrage (siehe expose)
% Komplizierte Abfragen / viele JOINs *(Bezug zu Aufbau Karte)
% Skalierbarkeit?
Ein Ansatz, welcher sich für kleine Online-Shops eignet, ist die Karten-Daten in einer relationalen Datenbank zu speichern \cite{johnson2013online}. Da ein Online-Shop nur einen Teil der Attribute einer Karte, wie Name, Seltenheit und Set, benötigt, kann das Datenbank Schema aus ein bis drei Tabellen bestehen, die diese Informationen speichern. Für eine komplexe Suchfunktion, die alle Attribute einer Karte berücksichtigt, eignet sich dieser Ansatz allerdings nicht, da hier viele Tabellen benötigt werden, um die Daten zu speichern. Bei Daten mit vielen Beziehungen untereinander sorgen Relationale Datenbanken für komplexe Abfragen mit vielen \verb|JOIN|-Befehlen, da diese hoch-strukturierte Daten nicht unterstützen \cite{robinsongraph:2015}. Des Weiteren sorgen viele \verb|JOIN|-Befehle in einer Abfrage dafür, dass diese schlecht skaliert. Dies ist in Neo4j nich der Fall \cite{robinsongraph:2015}. In \autoref{listing:complex.sql} ist beispielhaft eine komplexe Suchanfrage mit 7 benötigten \verb|JOIN|-Befehlen angegeben.

\begin{listing}[t]
    \caption{Komplexe SQL-Abfrage}
    \label{listing:complex.sql}
    \begin{lstlisting}[language=SQL]
SELECT * FROM `translations` AS `t` 
JOIN `cards` ON `cards`.`id` = `t`.`card_id`
JOIN `cards_in_set` ON `cards_in_set`.`card_id` = `t`.`card_id`
JOIN `types` ON `cards`.`type_id` = `types`.`id`
JOIN `card_colors` ON `cards`.`id` = `card_colors`.`card_id`
JOIN `card_has_ability` ON `cards`.`id` = `card_has_ability`.`card_id`
JOIN `keyword_abilities` ON `card_has_ability`.`ability_id` = `keyword_abilities`.`id`
JOIN `sets` ON `sets`.`id` = `cards_in_set`.`set_id`
WHERE
    `sets`.release > "2001-01-01" AND
    `cards_in_set`.`rarity` = "RARE" AND
    `t`.`lang` = "German" AND
    `card_colors`.color = "BLUE" AND
    `types`.`type` = "Creature" AND
    `keyword_abilities`.`name` = "Flying" AND
    `cards`.`cmc` <= 5
    \end{lstlisting}
\end{listing}


\subsection{NoSQL Datenbanken}
\ac{NoSQL} Datenbanken wie Key-Value-, Document- oder Column-oriented Stores eignen sich nicht für Daten mit vielen Verknüpfungen, da sie die Daten nicht verbunden speichern. Eine Ausnahme sind die Graph-Datenbanken, da sie die Daten als Graph speichern, unterstützen sie daher Daten mit vielen Beziehungen untereinander von vornherein  \cite{robinsongraph:2015}. Außerdem skalieren Graph-Datenbanken sehr gut, da sie immer nur auf einem Teil des Graphen agieren und nicht den ganzen Graph betrachten müssen \cite{robinsongraph:2015}. In \cite{jaiswal2013comparative} wurde außerdem gezeigt, dass für Abfragen mit mehr als einem \verb|JOIN|-Befehl oder verschachtelten Abfragen Neo4j Abfragesprache Cypher besser geeignet ist. Zu einem ähnlichen Ergebnis, dass Neo4j bei Strukturabfragen besser geeignet ist, kamen auch Vicknair et. al. in \cite{vicknair2010comparison}. Aufgrund der vielen Attribute einer Karte und den damit verbunden Verknüpfungen erscheint Neo4j als Datenbank gut geeignet zu sein.