% !TEX root = ../TUCthesis.tex
%************************************************
\chapter{Ergebnisse}\label{ch:ergebnisse}
%************************************************
\section{Validierung des Gesamtkonzeptes}

%%%%%%%%%%%%%%%%%%%%%%%%%%%%%%%%%%%%%%%%%%%%%%%%%%%%%%%%%%%%%%%%%%%%%%%%%%%%%%%%%%%%%%%%%%%%%%%%%%%%%
% Testfälle und Beziehung zu Problemen aus Kapitel 4
%
% Eine oftmals hilfreiche Form, die Inhalte dieses Kapitels zu veranschaulichen, ist eine Tabelle,
% welche die Requirements / die Probleme aus Kapitel 4 und die Tests / Ergebnisse aus Kapitel
% 5 zuordnet. Aus dieser Tabelle ist dann ersichtlich, welche Teile abgedeckt, getestet, nicht
% testbar, erwartungsgemäß, abweichend etc. sind.
%
%
% Gesetz der Großen Zahlen, Varianz, Standardabweichung
%%%%%%%%%%%%%%%%%%%%%%%%%%%%%%%%%%%%%%%%%%%%%%%%%%%%%%%%%%%%%%%%%%%%%%%%%%%%%%%%%%%%%%%%%%%%%%%%%%%%
\begin{table}[t]
    \caption{Erforderliche Testfälle} 
    \myfloatalign
    \begin{tabularx}{\textwidth}{lX}
        \toprule 
        \tableheadline{Anforderung} & \tableheadline{Testfall} \\ 
        \midrule 
        Kartensuche: Schlüsselwort-Fähigkeit & Suche alle Karten mit einer bestimmten Fähigkeit \\
        Kartensuche: Textsuche & Suche nach einer Zeichenkette die in einem Kartennamen vorkommt\\
        Kartensuche: Verknüpfungen & Suche anhand verschiedener Karten-Attribute \\
        Turnier: Matchup Analyse & Berechne das Matchup eines Deck-Typen \\
        \bottomrule 
    \end{tabularx}
    \label{tab:req.tests}
\end{table}
In \autoref{tab:req.tests} befindet sich eine Auflistung der Probleme die in \autoref{ch:analyse} beschrieben wurden und welche Testfälle erforderlich sich, um die Anforderung zu überprüfen.

Jeder Test wurde 1000mal sowohl auf Neo4j als auch auf MySQL ausgeführt, um den Einfluss von Messfehlern zu reduzieren \emph{(Gesetz der Großen Zahlen)}. Bei jedem Test wurde die Ausführungszeit in Millisekunden (ms) und der Speicherverbrauch in MByte gemessen. Um sicherzugehen, dass Caching oder Systemprozesse die Ergebnisse nicht beeinflussen, wurden die 10 längsten und kürzesten Zeiten verworfen. Die übriggebliebenen Werte wurden gemittelt und die Standardabweichung wurde berechnet, um zu schauen wie groß die Fehlergrenze ist. 

Um zu erfahren wie gut die beiden Datenbanken skalieren, wurden außerdem die Testfälle stückweise auf größeren Datensätzen ausgeführt. Diese Datensätze wurden zufällig generiert und basieren nicht auf echten Daten. Als Staffelung für die Tests wurden \emph{100.000, 250.000, 500.000 und 1.000.000} Karten/Turniere gewählt.

\section{Beschreibung und Motivation der Testfälle}
%% Tests in normaler Sprache:
%% zB: Testfall X: Suche nach allen Karte die aus Set Y sind und Kriterium Z erfüllen
%%  -> guter test, da viele Verknpüfungen

% BEISPIEL!
% Testfall 1: simple-keyword-ability-search -> "A simple search to get the first 300 cards with the keyword ability of 'Flying'"
% Kartensuche => viele JOINS

% Testfall 2: simple-artist-card-search -> "A simple search for the cards that contain the name 'Forest' and which are either drawn by 'Aleksi Briclot' or 'John Avon'"
% Kartensuche -> Suche LIKE / CONTAINS

% Testfall 3: matchup -> 
% Turnier Analyse

% Testfall 4: top 10 decks/players
% Deck Analyse

\subsection{Testfall 1: Schlüsselwort-Fähigkeit}
Eine einfache Suche, um die ersten 300 Karten mit dem Schlüsselwort-Fähigkeit \emph{Flying} zu erhalten. Wie in \autoref{fig:uml:schema} zu sehen ist, werden die Fähigkeiten in MySQL in einer eigenen Tabelle gespeichert. In Neo4j hingegen können diese als Attribut von \emph{Card} hinterlegt werden \emph{(siehe \autoref{fig:graph:schema})}, da Neo4j den Datentyp \emph{Array} unterstützt. Mit diesem Test lässt sich also vergleichen wie gut sich Arrays im Vergleich zu joins beim durchsuchen eignen.
    
\subsection{Testfall 2: Textsuche}
Eine einfache Suche nach den Karten, die den Namen \emph{Forest} enthalten und die entweder von \emph{Aleksi Briclot} oder \emph{John Avon} gezeichnet wurden. Da Karten anhand ihres Namens identifiziert werden, ist es sinnvoll zu testen wie gut sich eine Textsuche in den beiden Datenbanksystemen funktioniert.

\subsection{Testfall 3: Verknüpfungen}
Eine komplexe Suche nach Karten, welche bestimmte Kriterien erfüllen \emph{(siehe \ref{listing:complex.sql})}. Es wird nach Karten gesucht, die folgende Eigenschaften erfüllen:
\begin{itemize}
    \item in einem Set nach dem \emph{01.01.2001} erschienen
    \item als Seltenheit \emph{RARE} besitzen
    \item in \emph{deutscher} Sprache erschienen
    \item als Farbe \emph{Blau} haben
    \item vom Typ \emph{Kreatur} sind
    \item die Fähigkeit \emph{Flying} besitzen
    \item umgewandelte Manakosten von \emph{maximal 5} haben
 \end{itemize} 

\subsection{Testfall 4: Matchup Analyse}
Berechnung des Matchup für den Deck-Typ \emph{Bant}. Es werden alle Matches analysiert an denen ein Deck vom Typ \emph{Bant} beteiligt war und überprüft ob das Deck gewonnen oder verloren hat. Die Ergebnisse werden dann anhand der gegnerischen Deck-Typen gruppiert und geordnet. Insgesamt gibt es im Standard-Testfall 1000 Turniere, die jeweils zwischen 2 und 6 Matches enthalten.  


\section{Übersicht und Bewertung der erzielten Ergebnisse}
%%
%% Graphen! + Berwertung (erwaret/unwartet, weil | DB X besser geeignet weil)
%%
%TODO: n=1000
\subsection{Laufzeit}
Die gemessenen Laufzeiten samt ihrer Standardabweichung sind in \autoref{fig:plot.runtime} als Fehlerbalkendiagramm und in \autoref{tab:runtime} tabellarisch dargestellt.
Der große Unterschied in der Laufzeit bei \emph{Testfall 1} lässt sich vermutlich damit erklären, dass bei MySQL für die Keywords zwei Tabellen und damit 2 zusätzliche \emph{JOINS} benötigt werden. Bei Neo4j hingegen sind die Keywords direkt als Array in den Karten-Daten enthalten.
Allgemein lässt sich erkennen, dass Neo4j \emph{(mit Ausnahme von Test 3)} eine kürze Laufzeit als MySQL hat.

%TODO: Test 3 -> neo4j langsamer wegen array suche in colors, abilities und types (x in [])

\subsection{Skalierbarkeit}
%TODO

\begin{figure}[t]
    \myfloatalign
    \begin{tikzpicture} 
        \begin{axis}[
            width=\linewidth, % Scale the plot to \linewidth
            grid=major, % Display a grid
            grid style={dashed,gray!30}, % Set the style
            xlabel=Testfälle, % Set the labels
            ylabel=Zeit,
            use units,
            y unit=ms,
            legend style={at={(0.5,-0.2)},anchor=north}, % Put the legend below the plot             
            ybar,
            enlargelimits=0.15,
            symbolic x coords={Test 1, Test 2, Test 3, Test 4},
            xtick=data,
            %nodes near coords,
            %nodes near coords align={vertical}
        ]
            \addplot+[error bars/.cd,y dir=both,y explicit]
            table[x=Test,y=Neo4j,y error=Neo4j error,col sep=comma]{evaluation-results/testcases-runtime.n-1000.csv};
            
            \addplot+[error bars/.cd,y dir=both,y explicit]
            table[x=Test,y=MySQL,y error=MySQL error,col sep=comma]{evaluation-results/testcases-runtime.n-1000.csv};
            
            \legend{Neo4j, MySQL}
        \end{axis}
    \end{tikzpicture}
    \caption{Ausführungszeit der einzelnen Testfälle}
    \label{fig:plot.runtime}
\end{figure}

\begin{table}[t]
    \caption{Gemessene Laufzeiten der Testfälle} 
    \myfloatalign
    \csvreader[tabular=lrrrr,table head=\toprule Test & Neo4j (ms) & Neo4j STD (ms) & MySQL (ms) & MySQL STD (ms) \\\midrule,late after last line=\\\bottomrule]%
    {evaluation-results/testcases-runtime.n-1000.csv}{Test=\test,Neo4j=\neo,Neo4j error=\neostd,MySQL=\mysql,MySQL error=\mysqlstd}%
    {\test & \neo & \neostd & \mysql & \mysqlstd}%
    \label{tab:runtime}
\end{table}




\begin{figure}[t]
    \myfloatalign
    \begin{tikzpicture} 
    \begin{axis}[
        width=0.8*\linewidth, % Scale the plot to \linewidth
        grid=major, % Display a grid
        grid style={dashed,gray!30}, % Set the style
        xlabel=Größe Datensatz, % Set the labels
        ylabel=Zeit,
        use units,
        y unit=ms,
        legend pos=north west,
    ]
    \addplot+[color=blue, mark=*,error bars/.cd,y dir=both,y explicit]
    table[x=Size,y=Neo4j,y error=Neo4j error,col sep=comma]{evaluation-results/testcases-runtime.n-1000.testcase1.csv};
    
    \addplot+[color=red, mark=square,error bars/.cd,y dir=both,y explicit]
    table[x=Size,y=MySQL,y error=MySQL error,col sep=comma]{evaluation-results/testcases-runtime.n-1000.testcase1.csv};
    
    \legend{Neo4j, MySQL}
    \end{axis}
    \end{tikzpicture}
    \caption{Skalierung Testfall 1}
    \label{fig:plot.scaling.testcase1}
\end{figure}

\begin{figure}[t]
    \myfloatalign
    \begin{tikzpicture} 
    \begin{axis}[
        width=0.8*\linewidth, % Scale the plot to \linewidth
        grid=major, % Display a grid
        grid style={dashed,gray!30}, % Set the style
        xlabel=Größe Datensatz, % Set the labels
        ylabel=Zeit,
        use units,
        y unit=ms,
        legend pos=north west,
    ]
    \addplot+[color=blue, mark=*,error bars/.cd,y dir=both,y explicit]
    table[x=Size,y=Neo4j,y error=Neo4j error,col sep=comma]{evaluation-results/testcases-runtime.n-1000.testcase2.csv};
    
    \addplot+[color=red, mark=square,error bars/.cd,y dir=both,y explicit]
    table[x=Size,y=MySQL,y error=MySQL error,col sep=comma]{evaluation-results/testcases-runtime.n-1000.testcase2.csv};
    
    \legend{Neo4j, MySQL}
    \end{axis}
    \end{tikzpicture}
    \caption{Skalierung Testfall 2}
    \label{fig:plot.scaling.testcase2}
\end{figure}

\begin{figure}[t]
    \myfloatalign
    \begin{tikzpicture} 
    \begin{axis}[
        width=0.8*\linewidth, % Scale the plot to \linewidth
        grid=major, % Display a grid
        grid style={dashed,gray!30}, % Set the style
        xlabel=Größe Datensatz, % Set the labels
        ylabel=Zeit,
        use units,
        y unit=ms,
        legend pos=north west,
    ]
    \addplot+[color=blue, mark=*,error bars/.cd,y dir=both,y explicit]
    table[x=Size,y=Neo4j,y error=Neo4j error,col sep=comma]{evaluation-results/testcases-runtime.n-1000.testcase3.csv};
    
    \addplot+[color=red, mark=square,error bars/.cd,y dir=both,y explicit]
    table[x=Size,y=MySQL,y error=MySQL error,col sep=comma]{evaluation-results/testcases-runtime.n-1000.testcase3.csv};
    
    \legend{Neo4j, MySQL}
    \end{axis}
    \end{tikzpicture}
    \caption{Skalierung Testfall 3}
    \label{fig:plot.scaling.testcase3}
\end{figure}

\begin{figure}[t]
    \myfloatalign
    \begin{tikzpicture} 
    \begin{axis}[
        width=0.8*\linewidth, % Scale the plot to \linewidth
        grid=major, % Display a grid
        grid style={dashed,gray!30}, % Set the style
        xlabel=Größe Datensatz, % Set the labels
        ylabel=Zeit,
        use units,
        y unit=ms,
        legend pos=north west,
    ]
    \addplot+[color=blue, mark=*,error bars/.cd,y dir=both,y explicit]
    table[x=Size,y=Neo4j,y error=Neo4j error,col sep=comma]{evaluation-results/testcases-runtime.n-1000.testcase4.csv};
    
    \addplot+[color=red, mark=square,error bars/.cd,y dir=both,y explicit]
    table[x=Size,y=MySQL,y error=MySQL error,col sep=comma]{evaluation-results/testcases-runtime.n-1000.testcase4.csv};
    
    \legend{Neo4j, MySQL}
    \end{axis}
    \end{tikzpicture}
    \caption{Skalierung Testfall 4}
    \label{fig:plot.scaling.testcase4}
\end{figure}