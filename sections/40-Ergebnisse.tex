% !TEX root = ../TUCthesis.tex
%************************************************
\chapter{Ergebnisse}\label{ch:ergebnisse}
%************************************************
\section{Validierung des Gesamtkonzeptes}

%%%%%%%%%%%%%%%%%%%%%%%%%%%%%%%%%%%%%%%%%%%%%%%%%%%%%%%%%%%%%%%%%%%%%%%%%%%%%%%%%%%%%%%%%%%%%%%%%%%%%
% Testfälle und Beziehung zu Problemen aus Kapitel 3
%
% Eine oftmals hilfreiche Form, die Inhalte dieses Kapitels zu veranschaulichen, ist eine Tabelle,
% welche die Requirements / die Probleme aus Kapitel 3 und die Tests / Ergebnisse aus Kapitel
% 4 zuordnet. Aus dieser Tabelle ist dann ersichtlich, welche Teile abgedeckt, getestet, nicht
% testbar, erwartungsgemäß, abweichend etc. sind.
%
%
% Gesetz der Großen Zahlen, Varianz, Standardabweichung
%%%%%%%%%%%%%%%%%%%%%%%%%%%%%%%%%%%%%%%%%%%%%%%%%%%%%%%%%%%%%%%%%%%%%%%%%%%%%%%%%%%%%%%%%%%%%%%%%%%%
\begin{table}[t]
    \caption{Erforderliche Testfälle} 
    \myfloatalign
    \begin{tabularx}{\textwidth}{lX}
        \toprule 
        \tableheadline{Anforderung} & \tableheadline{Testfall} \\ 
        \midrule 
        Kartensuche: Schlüsselwort-Fähigkeit & Suche alle Karten mit einer bestimmten Fähigkeit \\
        Kartensuche: Textsuche & Suche nach einer Zeichenkette die in einem Kartennamen vorkommt\\
        Kartensuche: Verknüpfungen & Suche anhand verschiedener Karten-Attribute \\
        Turnier: Matchup Analyse & Berechne das Matchup eines Deck-Typen \\
        Turnier: Top 10 Decks & Berechne die 10 besten Decks \\
        \bottomrule 
    \end{tabularx}
    \label{tab:req.tests}
\end{table}
In \autoref{tab:req.tests} befindet sich eine Auflistung der Probleme die in \autoref{ch:analyse} beschrieben wurden und welche Testfälle erforderlich sich, um die Anforderung zu überprüfen.

Jeder Test wurde 1000mal sowohl auf Neo4j als auch auf MySQL/MariaDB ausgeführt, um den Einfluss von Messfehlern zu reduzieren \emph{(Gesetz der Großen Zahlen)}. Bei jedem Test wurde die Ausführungszeit in Millisekunden (ms) und der Speicherverbrauch in MByte gemessen. Um sicherzugehen, dass Caching oder Systemprozesse die Ergebnisse nicht beeinflussen, wurden die 10 längsten und kürzesten Zeiten verworfen. Die übriggebliebenen Werte wurden gemittelt und die Standardabweichung wurde berechnet, um zu schauen wie groß die Fehlergrenzen sind. 

Um zu erfahren wie gut die beiden Datenbanken skalieren, wurden außerdem die Testfälle stückweise auf größeren Datensätzen ausgeführt. Diese Datensätze wurden zufällig generiert und basieren nicht auf echten Daten. Als Staffelung für die Tests wurden \emph{100.000, 250.000, 500.000 und 1.000.000} Karten gewählt.

\section{Beschreibung und Motivation der Testfälle}
%% Tests in normaler Sprache:
%% zB: Testfall X: Suche nach allen Karte die aus Set Y sind und Kriterium Z erfüllen
%%  -> guter test, da viele Verknpüfungen

\subsection{Testfall 1: Schlüsselwort-Fähigkeit}
Eine einfache Suche, um die ersten 300 Karten mit dem Schlüsselwort-Fähigkeit \emph{Flying} zu erhalten. Wie in \autoref{fig:uml:schema} zu sehen ist, werden die Fähigkeiten in MySQL/MariaDB in einer eigenen Tabelle gespeichert. In Neo4j hingegen können diese als Attribut von \emph{Card} hinterlegt werden \emph{(siehe \autoref{fig:graph:schema})}, da Neo4j den Datentyp \emph{Array} unterstützt. Mit diesem Test lässt sich also vergleichen wie gut sich Arrays im Vergleich zu \verb|JOINS| beim durchsuchen eignen.
    
\subsection{Testfall 2: Textsuche}
Eine einfache Suche nach den Karten, die den Namen \emph{Forest} enthalten und die entweder von \emph{Aleksi Briclot} oder \emph{John Avon} gezeichnet wurden. Da Karten anhand ihres Namens identifiziert werden, ist es sinnvoll zu testen wie gut sich eine Textsuche in den beiden Datenbanksystemen funktioniert.

\subsection{Testfall 3: Verknüpfungen}
Eine komplexe Suche nach Karten, welche bestimmte Kriterien erfüllen \emph{(siehe \ref{listing:complex.sql})}. Es wird nach Karten gesucht, die folgende Eigenschaften erfüllen:
\begin{itemize}
    \item in einem Set nach dem \emph{01.01.2001} erschienen
    \item als Seltenheit \emph{RARE} besitzen
    \item in \emph{deutscher} Sprache erschienen
    \item als Farbe \emph{Blau} haben
    \item vom Typ \emph{Kreatur} sind
    \item die Fähigkeit \emph{Flying} besitzen
    \item umgewandelte Manakosten von \emph{maximal 5} haben
 \end{itemize} 

\subsection{Testfall 4: Matchup Analyse}
Berechnung des Matchup für den Deck-Typ \emph{Bant}. Es werden alle Matches analysiert an denen ein Deck vom Typ \emph{Bant} beteiligt war und überprüft ob das Deck gewonnen oder verloren hat. Die Ergebnisse werden dann anhand der gegnerischen Deck-Typen gruppiert und geordnet. Insgesamt gibt es im Standard-Testfall 1000 Turniere mit etwas über 6500 Decks, die jeweils zwischen 2 und 6 Matches enthalten.  

\subsection{Testfall 5: Top 10 Decks}
Für jedes Deck wird das Verhältnis zwischen den gewonnen und verlorenen Matches berechnet und dann der Größe nach sortiert und die ersten 10 Decks gewählt. Je höher der Wert ist, desto besser ist das Deck unabhängig von der Anzahl der gespielten Partien. Insgesamt gibt es im Standard-Testfall 1000 Turniere mit etwas über 6500 Decks, die jeweils zwischen 2 und 6 Matches enthalten. 

\section{Übersicht und Bewertung der erzielten Ergebnisse}
%%
%% Graphen! + Berwertung (erwaret/unwartet, weil | DB X besser geeignet weil)
%%
\subsection{Laufzeit}
Die gemessenen Laufzeiten samt ihrer Standardabweichung sind in \autoref{fig:plot.runtime} als Fehlerbalkendiagramm dargestellt. Wie zu erwarten war ist Neo4j in Testfall 4 schneller als die SQL-Abfrage mit \verb|JOINS|. Überraschenderweise war Neo4j auch in Testfall 2 etwas schneller als MySQL/MariaDB, obwohl aufgrund der langjährigen Optimierungen der Textsuche in MySQL/MariaDB zu erwarten war, dass diese schneller sei. Der Grund dafür, dass Neo4j in den Testfällen 1 und 3 langsamer ist als MySQL/MariaDB liegt vermutlich daran, dass als Datentyp Arrays für \emph{color}, \emph{abilities} und \emph{types} benutzt wurden. Anscheinend ist in Neo4j die Suche innerhalb eines Arrays langsamer als die Suche mit einem \verb|JOIN| in SQL.

\begin{figure}[t]
    \myfloatalign
    \begin{tikzpicture} 
        \begin{axis}[
            width=0.75*\linewidth, % Scale the plot to \linewidth
            grid=major, % Display a grid
            grid style={dashed,gray!30}, % Set the style
            xlabel=Testfälle, % Set the labels
            ylabel=Laufzeit,
            use units,
            y unit=ms,
            %legend style={at={(0.5,-0.2)},anchor=north}, % Put the legend below the plot        
            legend pos=north west,
            enlarge y limits={upper,value=0.15},
            ybar,
            symbolic x coords={Test 1, Test 2, Test 3, Test 4, Test 5},
            xtick=data,
            nodes near coords,
            every node near coord/.append style={rotate=90, anchor=west}
        ]
            \addplot+[error bars/.cd,y dir=both,y explicit]
            table[x=Test,y=Neo4j,y error=Neo4j error,col sep=comma]{evaluation-results/testcases-runtime.n-1000.csv};
            
            \addplot+[error bars/.cd,y dir=both,y explicit]
            table[x=Test,y=MySQL,y error=MySQL error,col sep=comma]{evaluation-results/testcases-runtime.n-1000.csv};
            
            \legend{Neo4j, MySQL/MariaDB}
        \end{axis}
    \end{tikzpicture}
    \caption{Ausführungszeit der einzelnen Testfälle}
    \label{fig:plot.runtime}
\end{figure}

%\begin{table}[t]
%    \caption{Gemessene Laufzeiten der Testfälle} 
%    \myfloatalign
%    \csvreader[tabular=lrrrr,table head=\toprule Test & Neo4j (ms) & Neo4j STD (ms) & MySQL (ms) & MySQL STD (ms) \\\midrule,late after last line=\\\bottomrule]%
%    {evaluation-results/testcases-runtime.n-1000.csv}{Test=\test,Neo4j=\neo,Neo4j error=\neostd,MySQL=\mysql,MySQL error=\mysqlstd}%
%    {\test & \neo & \neostd & \mysql & \mysqlstd}%
%    \label{tab:runtime}
%\end{table}

\subsection{Skalierbarkeit}
Wie in \autoref{fig:plot.scaling.testcase1} zu sehen ist, skalieren sowohl Neo4j als auch MySQL/MariaDB gut, aber auch hier zeigt sich, dass Neo4j in allen gemessenen Punkten langsamer ist. Die Steigung ist bei beiden Datenbanken ungefähr gleich. Die Vermutung, dass die Suche in Arrays langsamer ist als mit \verb|JOINS| bestätigt sich in Testfall 3.

\begin{figure}[t]
    \myfloatalign
    \begin{tikzpicture} 
    \begin{axis}[
    width=0.70*\linewidth, % Scale the plot to \linewidth
    grid=major, % Display a grid
    grid style={dashed,gray!30}, % Set the style
    xlabel=\# Karten, % Set the labels
    ylabel=Laufzeit,
    use units,
    y unit=ms,
    legend pos=north west,
    ]
    \addplot+[color=blue, mark=*,error bars/.cd,y dir=both,y explicit]
    table[x=Size,y=Neo4j,y error=Neo4j error,col sep=comma]{evaluation-results/testcases-runtime.n-1000.testcase1.csv};
    
    \addplot+[color=red, mark=square,error bars/.cd,y dir=both,y explicit]
    table[x=Size,y=MySQL,y error=MySQL error,col sep=comma]{evaluation-results/testcases-runtime.n-1000.testcase1.csv};
    
    \legend{Neo4j, MySQL/MariaDB}
    \end{axis}
    \end{tikzpicture}
    \caption{Skalierung Testfall 1}
    \label{fig:plot.scaling.testcase1}
\end{figure}

In Testfall 2 bestätigt sich, dass MySQL/MariaDB bei der Textsuche optimierter ist als Neo4j, auch wenn Neo4j mit dem normalen Datensatz ein wenig schneller ist. In \autoref{fig:plot.scaling.testcase2} ist zu sehen, dass sich auch hier mit Verdopplung der Karten die Laufzeit der Neo4j-Abfrage um den Faktor 1.5 - 2 steigt, wohingegen die Laufzeit bei MySQL/MariaDB nur um 5\% - 10\% pro gemessenem Punkt steigt. MySQL/MariaDB ist also wie erwartet besser für Textsuchen geeignet als Neo4j.

\begin{figure}[H]
    \myfloatalign
    \begin{tikzpicture} 
    \begin{axis}[
    width=0.70*\linewidth, % Scale the plot to \linewidth
    grid=major, % Display a grid
    grid style={dashed,gray!30}, % Set the style
    xlabel=\# Karten, % Set the labels
    ylabel=Laufzeit,
    use units,
    y unit=ms,
    legend pos=north west,
    ]
    \addplot+[color=blue, mark=*,error bars/.cd,y dir=both,y explicit]
    table[x=Size,y=Neo4j,y error=Neo4j error,col sep=comma]{evaluation-results/testcases-runtime.n-1000.testcase2.csv};
    
    \addplot+[color=red, mark=square,error bars/.cd,y dir=both,y explicit]
    table[x=Size,y=MySQL,y error=MySQL error,col sep=comma]{evaluation-results/testcases-runtime.n-1000.testcase2.csv};
    
    \legend{Neo4j, MySQL/MariaDB}
    \end{axis}
    \end{tikzpicture}
    \caption{Skalierung Testfall 2}
    \label{fig:plot.scaling.testcase2}
\end{figure}

Wie in  \autoref{fig:plot.scaling.testcase3} zu sehen ist, hat die Laufzeit bei Neo4j ein lineares Wachstum, wohingegen sie bei MySQL/MariaDB relativ konstant bleibt: bei doppelter Menge an Karten, verdoppelt sich \emph{(ungefähr)} die Laufzeit in Neo4j. Es bietet sich an zu überprüfen, ob mit einem anderen Schema, welches auf Arrays verzichtet, Neo4j eine bessere Laufzeit und Skalierbarkeit erreicht.
 
 \begin{figure}[t]
     \myfloatalign
     \begin{tikzpicture} 
     \begin{axis}[
     width=0.70*\linewidth, % Scale the plot to \linewidth
     grid=major, % Display a grid
     grid style={dashed,gray!30}, % Set the style
     xlabel=\# Karten, % Set the labels
     ylabel=Laufzeit,
     use units,
     y unit=ms,
     legend pos=north west,
     ]
     \addplot+[color=blue, mark=*,error bars/.cd,y dir=both,y explicit]
     table[x=Size,y=Neo4j,y error=Neo4j error,col sep=comma]{evaluation-results/testcases-runtime.n-1000.testcase3.csv};
     
     \addplot+[color=red, mark=square,error bars/.cd,y dir=both,y explicit]
     table[x=Size,y=MySQL,y error=MySQL error,col sep=comma]{evaluation-results/testcases-runtime.n-1000.testcase3.csv};
     
     \legend{Neo4j, MySQL/MariaDB}
     \end{axis}
     \end{tikzpicture}
     \caption{Skalierung Testfall 3}
     \label{fig:plot.scaling.testcase3}
    \end{figure}

In \autoref{fig:plot.scaling.testcase4} zeigt sich, dass die Laufzeit der SQL-Abfrage für Testfall 4 ein exponentielles Wachstum hat. Bei der Neo4j Abfrage hingegen steigt die Laufzeit linear, das heißt für Turnier/Matchup-Analysen ist Neo4j besser geeignet als MySQL/MariaDB. 

\begin{figure}[H]
    \myfloatalign
    \begin{tikzpicture} 
    \begin{axis}[
    width=0.70*\linewidth, % Scale the plot to \linewidth
    grid=major, % Display a grid
    grid style={dashed,gray!30}, % Set the style
    xlabel=\# Turniere, % Set the labels
    ylabel=Laufzeit,
    use units,
    y unit=ms,
    legend pos=north west,
    ]
    \addplot+[color=blue, mark=*,error bars/.cd,y dir=both,y explicit]
    table[x=Size,y=Neo4j,y error=Neo4j error,col sep=comma]{evaluation-results/testcases-runtime.n-1000.testcase4.csv};
    
    \addplot+[color=red, mark=square,error bars/.cd,y dir=both,y explicit]
    table[x=Size,y=MySQL,y error=MySQL error,col sep=comma]{evaluation-results/testcases-runtime.n-1000.testcase4.csv};
    
    \legend{Neo4j, MySQL/MariaDB}
    \end{axis}
    \end{tikzpicture}
    \caption{Skalierung Testfall 4}
    \label{fig:plot.scaling.testcase4}
\end{figure}

Allerdings gilt dies nicht für Testfall 5, denn \autoref{fig:plot.scaling.testcase5} zeigt, dass hier Neo4j langsamer ist als MySQL/MariaDB. Sowohl MySQL/MariaDB als auch Neo4j haben ein lineares Wachstum, auch wenn es bei Neo4j nach einem beschränkten Wachstum aussieht. Ein beschränktes Wachstum erscheint aber für die Laufzeit nicht sinnvoll. Des Weiteren hat Neo4j ein stärkeres Wachstum als MySQL/MariaDB. Die Tatsache, dass beide Datenbanken bei den Abfragen deutlich langsamer sind, ist in der Praxis vernachlässigbar. Es ist ausreichend die besten Decks nur einmal am Tag oder wenn ein neues Turnier hinzugefügt wurde zu berechnen und das Ergebnis zu speichern: wenn kein neues Turnier hinzugefügt wird, ändert sich auch nicht die Top 10 Liste. Aus diesem Grund muss die Abfrage nicht in Echtzeit geschehen, sondern kann durchaus einige Sekunden in Anspruch nehmen.


\begin{figure}[t]
    \myfloatalign
    \begin{tikzpicture} 
        \begin{axis}[
            width=0.70*\linewidth, % Scale the plot to \linewidth
            grid=major, % Display a grid
            grid style={dashed,gray!30}, % Set the style
            xlabel=\# Turniere, % Set the labels
            ylabel=Laufzeit,
            use units,
            y unit=ms,
            legend pos=north west,
        ]
        \addplot+[color=blue, mark=*,error bars/.cd,y dir=both,y explicit]
        table[x=Size,y=Neo4j,y error=Neo4j error,col sep=comma]{evaluation-results/testcases-runtime.n-1000.testcase5.csv};
        
        \addplot+[color=red, mark=square,error bars/.cd,y dir=both,y explicit]
        table[x=Size,y=MySQL,y error=MySQL error,col sep=comma]{evaluation-results/testcases-runtime.n-1000.testcase5.csv};
        
        \legend{Neo4j, MySQL/MariaDB}
        \end{axis}
    \end{tikzpicture}
    \caption{Skalierung Testfall 5}
    \label{fig:plot.scaling.testcase5}
\end{figure}


\subsection{Speicherverbrauch}
Der gemessene Speicherverbrauch in MByte unterscheidet sich zwischen MySQL/MariaDB und Neo4j faktisch nicht, das heißt er beträgt weniger als 0.1\%. Des Weiteren bleibt dieser auch bei der Skalierung der Datenbank relativ konstant mit $\pm1$MByte Abweichung. Aus diesem Grund wird hier nicht weiter auf den Speicherverbrauch eingegangen.